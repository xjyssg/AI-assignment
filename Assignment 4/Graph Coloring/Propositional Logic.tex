\documentclass[12pt, a4paper]{report}
\usepackage{graphicx}
\usepackage[super]{nth}
\usepackage{color} % Colour control
\definecolor{db}{cmyk}{1,0.5,0,0.5}
\usepackage[Glenn]{fncychap}
\usepackage{enumerate}
\usepackage{diagbox}

\setlength{\parindent}{0pt}

\title{
    \vspace{0.5cm} \textcolor{db}{\textsc{LINGI2261: Artificial Intelligence}} \\
    \vspace{0.5 cm} \rule{10 cm}{0.5pt} \\
    \vspace{0.5 cm} \Large{Assignment 3: Project: Adversarial Search (Part2)} \\
    \vspace{3 cm}
    \begin{flushright}
        \large
        \textbf{Groupe 81} \\
        Sanae \textsc{Abdelouassaa} \\
        Jiayue \textsc{Xue} \\     
    \end{flushright}
    \vspace{0.5 cm}
}
\begin{document}

\maketitle

\tableofcontents
\chapter{Propositional Logic}
\section{Models and Logical Connectives}
Directive: For each sentence, give its number of valid interpretations, i.e. the number of times the sentence is true (considering for each sentence all the proposition variables A, B, C and D).

Question 1:
\begin{equation}
    (\neg A \vee C) \wedge (\neg B \vee C)
\end{equation}
Answer:
\begin{table}[h]
    \centering
    \caption{Question 1}
    \begin{tabular}{|c|c|c|c|c|c|}
    \hline
    A & B & C & $\neg A \vee C$ & $\neg B \vee C$ & Result \\
    \hline
    true & true & true & true & true & true \\
    \hline
    true & true & false & false & false & false \\
    \hline
    true & false & true & true & true & true \\
    \hline
    true & false & false & false & true & false \\
    \hline
    false & true & true & true & true & true \\
    \hline
    false & true & false & true & false & false \\
    \hline
    false & false & true & true & true & true \\
    \hline
    false & false & false & true & true & true \\
    \hline
    \end{tabular}
\end{table}

Question 2:
\begin{equation}
    (C \Rightarrow \neg A) \wedge \neg (B \vee C)
\end{equation}
Answer:
\begin{table}[h]
    \centering
    \caption{Question 2}
    \begin{tabular}{|c|c|c|c|c|c|}
    \hline
    A & B & C & $C \Rightarrow \neg A$ & $B \vee C$ & Result \\
    \hline
    true & true & true & false & true & false \\
    \hline
    true & true & false & true & true & false \\
    \hline
    true & false & true & false & true & false \\
    \hline
    true & false & false & true & false & true \\
    \hline
    false & true & true & true & true & false \\
    \hline
    false & true & false & true & true & false \\
    \hline
    false & false & true & true & true & false \\
    \hline
    false & false & false & true & false & true \\
    \hline
    \end{tabular}
\end{table}

Question 3:
\begin{equation}
    (\neg A \vee B) \wedge \neg (B \Rightarrow \neg C) \wedge \neg (\neg D \Rightarrow A)
\end{equation}
Answer:
\begin{table}[h]
    \centering
    \caption{Question 3}
    \begin{tabular}{|c|c|c|c|c|c|c|c|}
    \hline
    A & B & C & D & $\neg A \vee B$ & $B \Rightarrow \neg C$ & $\neg D \Rightarrow A$ & Result \\
    \hline
    true & true & true & true & true & false & true & false \\
    \hline
    true & true & true & false & true & false & true & false \\
    \hline
    true & true & false & true & true & true & true & false \\
    \hline
    true & true & false & false & true & true & true & false \\
    \hline
    true & false & true & true & false & true & true & false \\
    \hline
    true & false & true & false & false & true & true & false \\
    \hline
    true & false & false & true & false & true & true & false \\
    \hline
    true & false & false & false & false & true & true & false \\
    \hline
    false & true & true & true & true & false & true & false \\
    \hline
    false & true & true & false & true & false & false & true \\
    \hline
    false & true & false & true & true & true & true & false \\
    \hline
    false & true & false & false & true & true & false & false \\
    \hline
    false & false & true & true & true & true & true & false \\
    \hline
    false & false & true & false & true & true & false & false \\
    \hline
    false & false & false & true & true & true & true & false \\
    \hline
    false & false & false & false & true & true & false & false \\
    \hline
    \end{tabular}
\end{table}


\section{Graph Coloring Problem}

Question 1: Explain how you can express this problem with propositional logic. What are the relations and how do you translate them?

Answer:
In the graph coloring problem, there are two constraints:
\begin{enumerate}
    \item 

\end{enumerate}
Question 2: Translate your model into Conjunctive Normal Form (CNF) and write it in your report.
Answer:


\end{document}